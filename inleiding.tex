\chapter{Inleiding}
\label{inleiding}
Het reconstrueren van fresco's waarvan in opgravingen fragmenten gevonden worden
is een moeilijke taak. Men kan het vergelijken met het oplossen van een enorme
puzzel waarvan de stukken arbitraire vormen hebben, de meesten hun originele
kleur zijn verloren en er vele anderen ontbreken. Daarbovenop komt nog eens dat
er heel wat stukken over de eeuwen heen een zekere vorm van slijtage hebben
ondervonden, waardoor ze niet meer perfect op elkaar passen en dus confirmatie nog moeilijker maken.\\

[afbeelding van enkele opgegraven stukken]\\

De stukken van de rand van het fresco met recht afgelijnde kanten of die nog een voldoende zichtbaar geometrisch patroon bevatten, zijn in vergelijking met de
anderen eenvoudig met elkaar te verbinden. De overige fragmenten die minder informatie bevatten zijn echter een nachtmerrie om aan elkaar te koppelen, gezien het menselijke visuele systeem niet
gemaakt is om dat soort grillige randen te vergelijken en aan elkaar te zetten. Om deze reden is het normaalgezien noodzakelijk om vele mogelijke kandidaten visueel
te onderscheiden en ze over elkaar te laten glijden om te zien of het past en waar precies.\\

In deze context situeert zich het \textbf{thera}\footnote{Thera is de oude naam voor het huidige griekse eiland genaamd Santorini, waar het project voor het eerst in de praktijk werd toegepast.} project, dat probeert om het werk van de archeoloog gemakkelijker te maken door middel van een software platform. De
redenering achter het project is dat wat voor een mens repititief en tijdsabsorberend zou zijn, geautomatiseerd zou kunnen worden m.b.v. een computer. Dergelijk systeem werd in 2007 aan de \emph{Princeton} universiteit in Amerika geconcipieerd. Sindsdien is er door verschillende onderzoekers van over de hele wereld aan gewerkt om alle noodzakelijke delen te ontwerpen, te implementeren en te integreren. [vermeld hoofdonderzoekers]. De werking van het systeem wordt in het 
volgende hoofdstuk nader toegelicht.\\

Deze thesis draait rond het maken van een uitbreiding op \'{e}\'{e}n van deze delen. De uitbreiding moet de gebruikers van het systeem in staat stellen om de beschikbare data op nieuwe manieren te gebruiken, te visualiseren, aan te passen en te delen met medeonderzoekers.\\

// TODO: vorige stuk gaat niet zozeer over samenwerking meer, herschrijf dit
mss? De gangbare methode van manueel puzzelen van v\'o\'or het thera project leende
zich wel tot een zekere vorm van samenwerking, op voorwaarde dat men op dezelfde
site aanwezig was. Verder moest alle betreffende informatie van commentaren tot
classificaties in een centrale plek bewaard worden zodat iedereen erbij kon. Dit ging van memobriefjes op de fragmenten zelf tot manueel
aangepaste elektronische documenten. Dankzij het thera systeem kon
deze manier van werken veranderd worden, alle relevante kennis kan elektronisch
opgeslagen worden. Er was echter tot op heden nog geen gemakkelijke
manier om deze kennis met andere archeologen te delen en al zeker niet om de bevindingen van verschillende onderzoekers te combineren. 
Dat is het kernprobleem dat deze thesis zal proberen uit de wereld te helpen.
Kortom: de nodige aanpassingen en nieuwe implementaties zullen gedaan worden om het oude systeem collaboratief te maken, zodat het beheren, delen en combineren van data gemakkelijk en intuitief wordt.\\
