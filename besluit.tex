\chapter{Besluit en toekomstige toepassingen}
\label{besluit}

Net zoals een gebroken fresco in feite een puzzel is, kan men het thera project zien als de som van vele delen die in elkaar passen. Dit thesisproject is bedoeld als een stuk dat een ander perspectief biedt op het geheel en het in staat moet stellen om meer en sneller resultaten te boeken. Het complementeert de bestaande aanpakken en zorgt ervoor dat de resultaten van de automatische paarherkenning nog nuttiger gebruikt kunnen worden. Omdat het validatiewerk noodzakelijk door mensen moet gebeuren, kan dit niet zonder meer opnieuw door een algoritme gedaan worden indien er iets misloopt. De in dit thesisproject geproduceerde componenten proberen er onder andere voor te zorgen dat dit zo min mogelijk voorkomt.\\

Op het vlak van ontginning van nuttige informatie met nieuwe visualisaties en nieuwe manieren om de juiste patronen te ontdekken zijn er natuurlijk nog steeds vele opportuniteiten. Want --- zoals opgemerkt in een recente paper over het thera project [citatie siggraph submission 2011] --- het vinden van de juiste paren is zoals zoeken naar een naald in een hooiberg. Met elke nieuwe toevoeging aan de mogelijkheden van het platform is er de kans dat deze een manier is om de hooiberg te verkleinen, door te lichten met X-stralen of gewoonweg op een grote krachtige magneet in een windtunnel te plaatsen. Naar deze laatste methode is iedereen natuurlijk op zoek. Tot dan is het zeker belangrijk dat men gemakkelijk kan experimenteren alsook bijhouden en opvragen welk deel van de berg reeds doorkamt is, waar de gevonden naaldrijke aders zitten en wat hun eigenschappen zijn. Wie weet zijn de naalden immers niet van metaal\ldots
 
\section{Toekomst}
Met dit project is ook een verdere stap gezet naar mobiele toepassingen. Het data-luik staat bijvoorbeeld toe om applicaties voor tablets te schrijven die beroep kunnen doen op een externe database om een groot deel van het zware werk over te nemen. Deze soort programma's kunnen het manueel verifi\"eren van paren sneller en aangenamer maken. De huidige werkwijze is als volgt: nadat er een resem waarschijnlijke paren zijn ge\"identificeerd, kan men de kisten met fragmenten uit de opslagruimte halen en nakijken welke er \'echt passen. Gezien de grote hoeveelheid brokstukken is het niet mogelijk om ze allemaal bij de hand te houden. Daardoor duurt het altijd even voor de gewenste fragmenten gevonden worden. In het slechtste geval wordt er gewerkt op een (krachtige) desktop. Hierdoor is het nodig is om ofwel de namen en locaties van de fragmenten te onthouden, of een heleboel afbeeldingen af te drukken. Na het fysisch testen van de fragmenten moet men dan terug naar de desktop om de bevindingen in te geven. Gelukkig behoort een laptop ook tot de mogelijkheden hoewel applicaties als Browsematches en Griphos niet bepaald licht zijn. Het performantieprobleem wordt natuurlijk reeds een deel verholpen door het invoegen van een externe database. Ook kan men nu gemakkelijker met meerdere mensen en laptops tegelijkertijd werken aan de validaties wegens de automatische synchronisatie. Een stap verder zou zijn om een tablet te gebruiken. Er zijn reeds in het verleden experimenten geweest binnen het thera project om aanraakgevoelige omgevingen te maken en de hoop is dat tesamen met de resultaten van deze thesis er in de toekomst iets concreets van gemaakt kan worden.
