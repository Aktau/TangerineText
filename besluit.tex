\chapter{Besluit}
\label{besluit}

Dit project heeft een lange weg afgelegd. De initi\"ele beschrijving van de thesisopdracht ging over het maken van (innovatieve) gebruikersomgevingen om te helpen met het vinden van passende fragmentparen. De kernvraag luidde: ``welke informatie is nodig om te kunnen zeggen of een paar wel of niet past? Hoe kan die weergegeven worden? Kan dit bijvoorbeeld voor 1000 paren tegelijkertijd?'' Het zoeken naar d\'e nieuwe weergave van fragmentparen die een gebruiker in staat zou stellen om op effici\"entere wijze naar de enkele correcte paren in een zee van incorrecte paren te zoeken leverde lange tijd niets op. Waarschijnlijk komt dit omdat het niet de juiste insteek is. Het bekijken van een enorme hoeveelheid aan paren tegelijkertijd Niettemin zou het kunnen dat er inderdaad een visualisatie bestaat die een optimale relevante informatiedensiteit bezit. Een voorbeeld van een dense voostelling is een intensiteitsmap van zoveel mogelijk fragmentparen (stel 1 pixel per paar) waar de intensiteit gelijkgesteld wordt aan de beoordeling van een automatische herkenner of een combinatie van meerdere herkenners en menselijke input (cfr. Machine Learning). Maar hier kan een archeoloog niet zoveel meer uit leren dan gewoon te sorteren op deze karakteristiek en in die volgorde paar na paar na te kijken. Als een karakteristiek werkelijk een onderscheid kan maken tussen correcte en niet-correcte paren zal dit in ieder geval moeten gebeuren. Dit soort dense visualisaties heeft dus vooral nut voor de ontwikkelaars van de karakteristieken zelf, zij kunnen een beter overzicht krijven van de verdeling binnen de set van alle paren. Het passen van fragmenten is een complexe taak, maar niet zo complex dat het onmogelijk lijkt om een getal te plaatsen op de relatieve ``goedheid'' van een paar. Het zoeken naar zo'n getal of karakteristiek is daarom op zich een waardevolle bezigheid. Dat was echter niet het onderwerp van deze thesis, maar van een andere die in dezelfde tijd aan de K.U.Leuven werd gemaakt binnen het thera project [referentie thesis yassine].\\

In het kort zou er dus een weergave moeten gevonden worden die:

\begin{itemize}
	\item Compact is
	\item De relevante informatie bevat voor het beslissen over de correctheid van een paar
	\item Niet sorteerbaar is (dan is het probleem opgelost)
	\item Menselijke verificatie nodig heeft
\end{itemize}

Een goed voorbeeld van een voorstelling die compacter is dan het voorstellen van de paren zelf en niet sorteerbaar is, is het weergeven van de doorsnede. 

Vele evidente en minder evidente weergaven waren reeds te vinden in het thera project. Men kon de fragmenten op allerlei manieren bekijken: behalve de gewone weergave in kleur kan men ook de achterzijde bekijken, de randen alleen, de dikte van de fragmenten, zelfs de normalen van het oppervlak konden gevisualiseerd worden om een beter idee te krijgen van het reli\"ef (zo kunnen onder andere krassen en borstelafdrukken zichtbaarder worden). Tevens kon de doorsnede van de plaats waar beide fragmenten elkaar raken bekeken worden, dit bleek zo nuttig te zijn dat het in Browsematches standaard aan de linkerkant van een paar werd weergegeven. E\'en mogelijke uitwerking van de thesis zou dus geweest zijn nog een voorstelling te vinden die snelle en accurate beslissingen toelaat en bij voorkeur compact is. Helaas is het op dit vlak qua idee\"en nooit verder gekomen dan incrementele verbeteringen op de huidige resultaten (bvb. niet-lineaire doorsneden van fragment visualiseren). Andere beloftevolle insteken, zoals het gebruiken van contextinformatie en menselijke input om een virtuele classificator te trainen worden reeds onderzocht door onderzoekers Antonio Garcia Castaneda (Clusters) en Tom Funkhouser (Machine Learning) respectievelijk.\\

Geen revolutionaire idee\"en voor visualisaties die een thesis kunnen vullen en een heleboel reeds lopende projecten op alle vlakken, wat nu gedaan? Uit een kijk op de defici\"enties van het thera project bleek dat er aan nieuwe delen werd gewerkt maar de infrastructuur niet optimaal was om alles aan elkaar te knopen, om de resultaten bij de gebruiker te brengen. Het project klaarmaken voor een eventueel nieuw paradigma en het toegankelijker maken van de informatie die reeds beschikbaar was, werden het nieuwe doel.\\

Net zoals een gebroken fresco in feite een puzzel is, kan men het thera project zien als de som van vele delen die in elkaar passen. Dit thesisproject is bedoeld als een stuk dat een ander perspectief biedt op het geheel en het in staat moet stellen om meer en sneller resultaten te boeken. Het complementeert de bestaande aanpakken en zorgt ervoor dat de resultaten van de automatische paarherkenning nog nuttiger gebruikt kunnen worden. Omdat het finale validatiewerk noodzakelijk door mensen moet gebeuren, is het geproduceerde werk waardevol: het kan niet zonder meer opnieuw door een algoritme gegenereerd kan worden. De voor dit thesisproject gemaakte componenten proberen er onder andere voor te zorgen dat het verlies van informatie zo min mogelijk voorkomt door robuuste dataopslag en synchronisatie mogelijk te maken.\\

Op het vlak van ontginning van nuttige informatie met nieuwe visualisaties en nieuwe manieren om de juiste patronen te ontdekken zijn er natuurlijk nog steeds vele opportuniteiten. Want --- zoals opgemerkt in een recente paper over het thera project [citatie siggraph submission 2011] --- het vinden van de juiste paren is zoals zoeken naar een naald in een hooiberg. Met elke nieuwe toevoeging aan de mogelijkheden van het platform is er de kans dat deze een manier is om de hooiberg te verkleinen, door te lichten met X-stralen of gewoonweg op een grote krachtige magneet in een windtunnel te plaatsen. Naar deze laatste methode is iedereen natuurlijk op zoek. Tot dan is het zeker belangrijk dat men gemakkelijk kan experimenteren alsook bijhouden en opvragen welk deel van de berg reeds doorkamt is, waar de gevonden naaldrijke aders zitten en wat hun eigenschappen zijn. Misschien zijn de naalden immers niet van metaal\ldots