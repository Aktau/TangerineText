\chapter{Doelen \& Motivatie}
\label{hoofdstuk:doelen}

Het hoofddoel van het hele thera project is om het hele proces van fresco's reconstrueren zo gemakkelijk mogelijk te maken. Er moet een manier gevonden worden om een waardevolle contributie te maken aan het thera-ecosysteem zodat het zijn doel beter kan vervullen. Dit is onder andere mogelijk door de gebreken in de vorige oplossingen te proberen verhelpen alsook door nieuwe functionaliteiten te voorzien.\\

Gezien de huidige stand van zaken besproken in hoofstuk \ref{hoofdstuk:overzicht}, valt het op dat er nog behoorlijk wat zaken kunnen toegevoegd worden op het gebied van informatie en het visualiseren ervan. De ongelooflijke hoeveelheid data die het thera project reeds heeft geproduceerd en blijft produceren kan op een betere manier behandeld worden zodat het ware potentieel ervan naar de oppervlakte komt. De laatste stap in het sterk geautomatiseerde virtuele reconstructie proces wordt gekenmerkt door een nood aan interactie met de mens die elk soort hulpmiddel moet krijgen om een beslissing zo correct en snel mogelijk te maken. Natuurlijk gaat er niets boven de feitelijke fragmenten fysisch vastnemen en ze aan elkaar proberen te zetten, maar gezien dit een zeer tijdsrovende bezigheid is, moet dit zo veel mogelijk beperkt worden.\\

Hieronder beschreven staan verschillende deelaspecten waarop dit thesisproject tracht verbeteringen te maken. Bij elk aspect staat een beschrijving van wat het resultaat allemaal zou moeten mogelijk maken. Elk van deze aspecten kan men desgewenst onafhankelijk bekijken maar bijna altijd staan steunen ze op elkaar om werkelijk tot hun recht te komen. Wat is een visualisatie zonder een datamodel dat de juiste data op het juiste moment kan aanleveren? Wat is een uitgebreid databeheersysteem zonder een mogelijkheid om eigenlijk iets met deze data te doen? Om alle verbeteringen en nieuwe idee\"en te bundelen en te testen is er ook een applicatie gemaakt die dient om reeds een voorproef te geven van wat er mogelijk is met de ontwikkelde technologie.  

\section{Integratie}
Zoals eerder vermeld staat dit thesisproject niet alleen maar maakt het deel uit van een groter geheel. Binnen de grenzen van het mogelijke zou er moeten rekening gehouden worden met de integratie van de geschreven code met die van de andere onderzoekers. Dit verhoogt de kans dat het werk aanvaard wordt en ingang vindt in andere subprojecten. 

\section{Collaboratie}
Ideaal gezien zouden archeologen steeds toegang moeten krijgen tot hun project vanop elk gegeven moment vanop elke plaats in de wereld. Een mogelijk scenario hierbij is dat een ervaren archeoloog in de Verenigde Staten gevraagd wordt om zijn opinie te geven over de huidige stand van zaken (reeds ge\"identificeerde correcte paren, moeilijke gevallen, \ldots). Alle aanpassingen en commentaren die hij maakt worden automatisch ingevoegd en centraal beschikbaar gesteld voor de onderzoekers ter plekke. Op termijn moet het bijvoorbeeld zelfs mogelijk worden om amateurs te laten kijken naar de voorstellen en hun beoordeling te gebruiken om de nog na te kijken voorstellen te rangschikken.

\section{Gebruiksvriendelijkheid}
Archeologen zijn dikwijls niet opgeleid om complexe applicaties zonder meer te gebruiken. Om deze reden moet er moeite gestoken worden in het ontwerp van een visueel aangename en intuiti\"eve gebruikservaring. Om deze reden wordt in het programma steeds de aandacht gevestigd op hetgeen het belangrijkst is: de fragmenten en de paren. Bij voorkeur moet elke operatie gemakkelijk ontdekbaar zijn in de context waar ze kan gebruikt worden, met een woordje uitleg erbij.\\

Volgens vele studies op het gebied van gebruikersinterfaces [refereer een goed boek] geraken gebruikers gefrustreerd vanaf een operatie een zekere tijd duurt. Deze frustratiedrempel hangt een af van de aard van de operatie (het resultaat), de frequentie waarmee die uitgevoerd wordt, of er visuele tekenen van voortgang zijn en of er tijdens het wachten (steeds) iets anders kan uitgevoerd worden. Om deze reden is het van het grootste belang dit aspect in acht te nemen bij het ontwikkelen van de applicatie. Een algemene vaststelling: de tijd die een operatie mag innemen is omgekeerd evenredig met de frequentie waarmee deze operatie moet uitgevoerd worden. Deze regel in acht nemend is het duidelijk dat bijvoorbeeld het inladen van een scherm vol voorstellen, het veranderen van een attribuut, het filteren en sorteren en dergelijke meer acties zijn die met de grootst mogelijke snelheid moeten worden uitgevoerd. Hoe functioneel ook, een programma dat sloom reageert en elke computer op z'n knie\"en dwingt zal gretig vervangen worden door een applicatie die dat niet doet.\\

Voor al deze operaties is sinds het eerste ontwerp van de applicatie en het onderliggende raamwerk rekening gehouden met de implicaties van elke beslissing op de snelheid.

// hoort mss bij [DESIGN]
De gebruikersinterface van de oude programma's was goed en kon effici\"ent gebruikt worden mits enige training. Maar voor het ondersteunen van collaboratief werken moeten er natuurlijk allerhande nieuwe operaties toegevoegd
worden. Tijdens het implementatieproces werd het duidelijk dat de reeds bestaande code van het Browsematches niet uitbreidbaar genoeg was en die van Griphos te complex en belangrijk. Daardoor werd de beslissing genomen om de
basisinterface van Browsematches over te nemen maar alle onderliggende code te herschrijven zodat die uitbreidbaar zou zijn. Bij het opnieuw construeren van dit alles zijn er een aantal verbeteringen gebeurd die niet meteen te maken hebben met het collaboratie aspect maar wel met de workflow van het classificeren. Dit werd gedaan om het hele programma gebruiksvriendelijker en krachtiger te maken in functie van het hoofddoel van het project.

\section{Uitbreidbaarheid}
Het samenstellen van werken uit de oudheid is een zeer vakkennis- en ervaringsintensief proces. Hoewel men luistert naar wat de archeologen hierover te vertellen hebben --- zoals wat ze graag zouden zien of kunnen doen --- zijn er vele zaken die nu nog niet duidelijk zijn maar in de toekomst zeker aan het licht zullen komen. Dit kan bijvoorbeeld zijn omdat de onderzoekers in kwestie niet goed kunnen uitleggen waar ze naar kijken of hoe ze zoeken: na zovele jaren vertrouwen ze op hun moeilijk te defini\"eren intu\"itie. Anderzijds is het vertalen van het proces naar de computer een grotendeels onbetreden pad. Dit betekent dat een goede werkwijze in de realiteit misschien niet effici\"ent meer is als men de transitie maakt naar virtueel reconstrueren (zoals bij Griphos). Daarbovenop zijn er nog vele kansen om innovatieve nieuwe technieken aan te wenden die niet werkbaar zijn als men enkel over fysische fragmenten.\\

Om deze reden is het reeds van in het begin duidelijk dat het platform zal moeten vervangen of herbouwd worden als het niet met uitbreidbaarheid in gedachte ontworpen wordt. Er moeten moeiteloos nieuwe delen aan de applicatie en de onderliggende lagen kunnen toegevoegd worden om snel nieuwe idee\"en te incorporeren. Dit alles moet liefst mogelijk zijn zonder de ervaring van gebruikers met oudere versies of andere gebruikersinterfaces te degraderen.

\section{Data}
Van cruciaal belang is dat alle verzamelde data (zoals de classificatie van voorstellen) op een robuuste manier opgeslagen, gedeeld en ge\"incorporeerd kan worden. De toegang naar deze informatie moet effici\"ent en eenvoudig doorzoekbaar zijn. De huidige oplossingen zijn hiervoor ontoereikend en traag, zoals besproken in hoofdstuk \ref{hoofdstuk:overzicht}. Opdat het mogelijk zou zijn om onderzoekers aan hun eigen kopie\"en van het voorstellenbestand te laten werken en die op een later tijdsstip te synchroniseren, moet er ook een samenvoegingsprocedure mogelijk zijn\footnote{Dit is naar het model van de zogenaamde \emph{Distributed Version Control System (DCVS)} systemen zoals Git, Mercurial, \ldots}. Dergelijk systeem is ook nuttig voor het maken van pocketversies en in gebieden waar de internetconnectiviteit niet adequaat of onbestaande is. Belangrijk is dat er steeds een manier is om waardevolle data te combineren en aan te vullen zodat niets verloren gaat.\\

%http://en.wikipedia.org/wiki/Three-way_merge#Three-way_merge <--- (bij design\ldots)

\subsection{Uitbreidbaarheid}
De eerder besproken uitbreidbaarheid die nodig is manifesteert zich op het niveau van de data bijvoorbeeld zo: een onderzoeker vindt een nieuw algoritme om fragmentparen te rangschikken of men wil informatie over het dikteverschil tussen twee fragmenten opslaan. Idealiter zouden deze zaken als een attribuut moeten kunnen toegevoegd worden zodat elke gebruiker er op kan zoeken en sorteren zonder iets extra te hoeven installeren.\\
 
Enkel data die op geen enkele manier om te vormen valt naar een attribuut en dus een irreguliere vorm heeft zal een speciale module vereisen om te kunnen gebruiken. Een vereiste is natuurlijk wel dat dit geen effect mag hebben op de delen van het platform die hier geen weet van hebben.
 
\section{Visualisatie}
Er zijn vele visualisatiemanieren denkbaar die elkaar kunnen aanvullen bij het volbrengen van het zoek- en classificeerprocess. Zo stelt men zich bij de uitdrukking ``fresco's samenstellen'' waarschijnlijk een grote puzzel voor waar men ten alle tijde het overzicht kan behouden en stukken proberen te passen.\footnote{Het Griphos programma ging uit van het idee van kleine beheersbare stukken van de reuzenpuzzel (elk tafelblad zou een verzameling kunnen zijn van stukken die gerelateerd waren, bijvoorbeeld door hun vindplaats)} Deze puzzel visualisatie is visueel aantrekkelijk en biedt het menselijke patroonherkenningsvermogen\footnote{Iets waar computers de mens nog altijd niet in evenaren. Een ander thera subproject onderzoekt wel de mogelijkheid om zogenaamde 'clusters' te identificeren en te gebruiken voor reconstructie. [CITATIE]} de mogelijkheid om zich van zijn beste kant te laten zien. Echter, door de grote hoeveelheid aan fragmenten en dus mogelijke paren is het moeilijk om hier aan te beginnnen. Maar, hoe meer reeds geconfirmeerde paren er zijn, hoe duidelijker het globale beeld kan worden. Dit staat toe om een overzicht te krijgen van de vooruitgang en gerichter te zoeken naar stukken die nog ontbreken. Dit is een voorbeeld van een macro-perspectief.\\

Even naar het stuk waar de computer w\'el goed in is: fragmenten aan elkaar passen en rangschikken naar kans/overeenkomst/et cetera. Door een gemakkelijk navigeerbare lijst op te stellen van alle voorstellen die de reconstructie algoritmes hebben gedaan, kunnen er snel op elkaar passende fragmenten ge\"identificeerd worden. Dit kan men zien als een micro-perspectief of \emph{bottom-up} manier om fresco's te reconstrueren. Het is ook de aanpak die in Browsematches gebruikt wordt. Nadat men bijvoorbeeld met deze manier een deel acceptabele paren heeft ge\'identificeerd, kunnen deze bijvoorbeeld weergegeven worden op de grote puzzel en dienen zij als beginpunt. Dit maakt het globale beeld veel informatiever: stel dat duidelijke "gaten" ontstaan in een resem goede fragmentparen, dan kan er gericht gezocht worden naar een fragment dat erin past door te zoeken naar een fragment dat met elk van deze insluiters past (indien het gevonden werd bij de opgraving).\\

Kortom, het is duidelijk dat een visualisatie die in alle gevallen de meest geschikte is niet bestaat. De beschikbare informatie moet soms gewoon op andere manieren worden weergegeven. Om deze reden is het wenselijk om het mogelijk te maken snel nieuwe visualisaties in te bouwen die kunnen communiceren met andere delen van de applicatie en eventueel de informatie manipuleren. 