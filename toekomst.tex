\chapter{Toekomstig werk}
\label{toekomst}

Met dit project is ook een verdere stap gezet naar mobiele toepassingen. Het data-luik staat bijvoorbeeld toe om applicaties voor tablets te schrijven die beroep kunnen doen op een externe database om een groot deel van het zware werk over te nemen. Deze soort programma's kunnen het manueel verifi\"eren van paren sneller en aangenamer maken. De huidige werkwijze is als volgt: nadat er een resem waarschijnlijke paren zijn ge\"identificeerd, kan men de kisten met fragmenten uit de opslagruimte halen en nakijken welke er \'echt passen. Gezien de grote hoeveelheid brokstukken is het niet mogelijk om ze allemaal bij de hand te houden. Daardoor duurt het altijd even voor de gewenste fragmenten gevonden worden. In het slechtste geval wordt er gewerkt op een (krachtige) desktop. Hierdoor is het nodig is om ofwel de namen en locaties van de fragmenten te onthouden, of een heleboel afbeeldingen af te drukken. Na het fysisch testen van de fragmenten moet men dan terug naar de desktop om de bevindingen in te geven. Gelukkig behoort een laptop ook tot de mogelijkheden hoewel applicaties als Browsematches en Griphos niet bepaald licht zijn. Het performantieprobleem wordt natuurlijk reeds een deel verholpen door het invoegen van een externe database. Ook kan men nu gemakkelijker met meerdere mensen en laptops tegelijkertijd werken aan de validaties wegens de automatische synchronisatie. Een stap verder zou zijn om een tablet te gebruiken. Er zijn reeds in het verleden experimenten geweest binnen het thera project om aanraakgevoelige omgevingen te maken en de hoop is dat tesamen met de resultaten van deze thesis er in de toekomst iets concreets van gemaakt kan worden.

Het versturen van afbeeldingen over het netwerk! 3D-modellen? render? (hoe dit efficient te doen? niet opnieuw uitvinden van het wiel, gebruik een goede codec)